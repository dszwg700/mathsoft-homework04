\documentclass[12pt,a4paper]{ctexart}
\usepackage{amsmath}
\usepackage{graphicx}
\usepackage{xltxtra}
\newtheorem{theorem}{Theorem}
\usepackage{algorithm}
\usepackage{url}
\usepackage{cite}
\usepackage{algorithmicx}
\usepackage{algpseudocode}
\usepackage{listings}
\usepackage{color}
\usepackage{subfigure}
\usepackage{caption}
\usepackage[hmargin={3.18cm, 3.18cm}, width=14.64cm, vmargin={2.54cm, 2.54cm}, height=24.62cm]{geometry}

\renewcommand{\algorithmicrequire}{\textbf{Input:}}  
\renewcommand{\algorithmicensure}{\textbf{Output:}}  

\definecolor{mygreen}{rgb}{0,0.6,0}
\definecolor{mygray}{rgb}{0.5,0.5,0.5}
\definecolor{mymauve}{rgb}{0.58,0,0.82}

\lstset{ %
  backgroundcolor=\color{white},   % choose the background color
  basicstyle=\footnotesize,        % size of fonts used for the code
  breaklines=true,                 % automatic line breaking only at whitespace
  captionpos=b,                    % sets the caption-position to bottom
  commentstyle=\color{mygreen},    % comment style
  escapeinside={\%*}{*)},          % if you want to add LaTeX within your code
  keywordstyle=\color{blue},       % keyword style
  stringstyle=\color{mymauve},     % string literal style
}

\makeatletter
\newenvironment{breakablealgorithm}
  {% \begin{breakablealgorithm}
   \begin{center}
     \refstepcounter{algorithm}% New algorithm
     \hrule height.8pt depth0pt \kern2pt% \@fs@pre for \@fs@ruled
     \renewcommand{\caption}[2][\relax]{% Make a new \caption
       {\raggedright\textbf{\ALG@name~\thealgorithm} ##2\par}%
       \ifx\relax##1\relax % #1 is \relax
         \addcontentsline{loa}{algorithm}{\protect\numberline{\thealgorithm}##2}%
       \else % #1 is not \relax
         \addcontentsline{loa}{algorithm}{\protect\numberline{\thealgorithm}##1}%
       \fi
       \kern2pt\hrule\kern2pt
     }
  }{% \end{breakablealgorithm}
     \kern2pt\hrule\relax% \@fs@post for \@fs@ruled
   \end{center}
  }
\makeatother

\title{Mandelbrot set的生成与探索}
\author{明安图 \\ 统计学 319101841}
\date{}
\bibliographystyle{plain}
\begin{document}
\maketitle

\begin{abstract}
  Mandelbrot集是使得函数$f(z)=z^2+c$从$z=0$时开始迭代不会趋于无穷的复数$c$的集合. 本文基于bmp位图提供了一种Mandelbrot set可视化算法以及具体实现方法
  \par\textbf{关键词:}Mandelbrot set
\end{abstract}

\section{引言}
Mandelbrot集的图像表现出一个精心设计的无限复杂的边界,在放大倍率增加的情况下,它揭示了越来越精细的递归细节;在数学上,人们会说曼德布洛特集合的边界是分形曲线.此递归细节的“样式”取决于所检查的集合边界的区域.曼德布洛特集合图像可以通过对复数进行采样和测试来创建,用于每个采样点$c$,序列$f(0),f(f(0))\cdots$是否趋于无穷.
\section{问题背景介绍}
Mandekbrot set起源于复杂动力学,法国数学家皮埃尔·法图(Pierre Fatou)和加斯顿·朱莉娅(Gaston Julia)在20世纪初首次研究了这一领域。这个分形于1978年由Robert W. Brooks和Peter Matelski首次定义和绘制,作为对Kleinian群研究的一部分\cite{wikimandelbrot}.Mandelbrot集是复平面上$c$值的集合,对于这个集合,二次映射迭代下的临界点轨道
$$
z_{n+1}=z^2_n+c
$$
保持有界.即如果$c$属于Mandelbrot集,当且仅当对于所有$n\geq 0,|z_n| \leq 2$.
\section{数学理论}
Mandelbrot set基于以下定理\cite{mandelbrotsetweb}:
\begin{theorem}
  0的轨道趋于无穷当且仅当它在某些点上的模大于2
\end{theorem}
该定理特定于$z \rightarrow z^2+c$,但可以通过改变阈值2来适应其他多项式族.算法伪代码如下:
\begin{breakablealgorithm}
\caption{Mandelbrot set algorithm}\label{alg:cap}

\begin{algorithmic}
  \Require Iteration $N$
  \Ensure Mandelbrot set
  \State $z \leftarrow 0$

  \For {each pixel p$(a,b)$ of the image}
  \State $c \leftarrow a+bi$
  \For {$i=1 \rightarrow N$}
   \If {$|z|>2$}
    \State $p \leftarrow white$
    \State go to the next pixel
    \Else
    \State $z \rightarrow z^2+c$
   \EndIf
  \EndFor
   
  \If {$z$ reached its natural end}
  \State $p \leftarrow black$
  \State go to the next pixel
  \EndIf
 \EndFor
\end{algorithmic}
\end{breakablealgorithm}

\section{数值算例}
迭代100次后的图像
\begin{figure}[H]
  \subfigure[全图]{
    \includegraphics[width=.5\textwidth]{01.bmp}}
   \hspace{0.5in}
   \subfigure[局部]{
     \includegraphics[width=.5\textwidth]{02.bmp}}
   \caption{迭代100次}
      
\end{figure}
迭代20次的图像
\begin{figure}[H]
  \subfigure[全图]{
    \includegraphics[width=.5\textwidth]{03.bmp}}
  \hspace{0.5in}
  \subfigure[局部]{
    \includegraphics[width=.5\textwidth]{04.bmp}}
  \caption{迭代20次}
\end{figure}

\section{结论}
比较上面的结果,可以发现,随着迭代次数的增加,图像的光滑程度下降了,边界上的小图与主图的连接部分变窄.说明Mandelbrot集在当迭代次数趋于无穷时,各个部分之间是完全相切的.
\bibliography{reference}
  
\end{document}
